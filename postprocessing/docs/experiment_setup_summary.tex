\documentclass[a4paper,11pt]{article}
\usepackage[utf8]{inputenc}
\usepackage{geometry}
\usepackage{hyperref}
\usepackage{listings}
\usepackage{xcolor}

\geometry{margin=2.5cm}

\title{Riassunto Setup Esperimento TIDES}
\author{Filippo Di Ludovico}
\date{\today}

\begin{document}

\maketitle

\section{Introduzione}
Questo documento riassume i passaggi eseguiti per configurare l'esperimento \texttt{TIDES}, mirato alla simulazione di onde di Kelvin in un bacino idealizzato (Adriatico).

\section{Struttura delle Directory}
\begin{itemize}
    \item \textbf{NEMO Root}: \texttt{/home/STUDENTI/filippo.diludovico/nlo/nemo\_4.2.0}
    \item \textbf{Configurazione}: \texttt{cfgs/TIDES} (copiata da \texttt{GYRE\_PISCES})
    \item \textbf{Directory di lavoro}: \texttt{.../Uni/NLO/03\_tides/TIDES/EXP00/tmp\_work}
\end{itemize}

\section{File Modificati}
Di seguito il dettaglio delle modifiche apportate in \texttt{MY\_SRC}.

\subsection{usrdef\_hgr.F90 (Griglia Orizzontale)}
Modificato per definire il dominio Adriatico idealizzato.
\begin{itemize}
    \item \textbf{Origine}: 16°E, 40°N
    \item \textbf{Risoluzione}: $\sim 10$ km (\texttt{nn\_GYRE = 1})
    \item \textbf{Rotazione}: Nessuna ($\sin \alpha = 0, \cos \alpha = 1$)
\end{itemize}

\begin{lstlisting}[language=Fortran, frame=single, basicstyle=\small\ttfamily]
zlam1 = 16._wp
zphi1 = 40._wp
ze1 = 10000._wp / REAL( nn_GYRE , wp )
\end{lstlisting}

\subsection{usrdef\_istate.F90 (Stato Iniziale)}
Adattato da \texttt{TSUNAMI} per generare una perturbazione Gaussiana dell'SSH.
\begin{itemize}
    \item \textbf{Centro}: 19.0°E, 40.5°N (Canale d'Otranto, più a Ovest per evitare confini)
    \item \textbf{Raggio ($R$)}: 20 km (aggiornato da 50km per segnale più pulito)
    \item \textbf{Ampiezza ($A_0$)}: Variabile (0.05m, 0.5m, 1.0m)
\end{itemize}

\subsection{usrdef\_nam.F90 (Dimensioni)}
Dimensioni del dominio impostate per un rettangolo allungato:
\begin{lstlisting}[language=Fortran, frame=single, basicstyle=\small\ttfamily]
kpi = 20 * nn_GYRE + 2  ! Larghezza (x)
kpj = 100 * nn_GYRE + 2 ! Lunghezza (y)
\end{lstlisting}

\subsection{usrdef\_sbc.F90 (Forcing)}
\textbf{Vento Rimosso}: Tutti i termini di stress del vento (\texttt{utau, vtau}) e moduli (\texttt{wndm}) sono stati forzati a 0.0 per isolare la dinamica d'onda.

\section{Aggiornamenti e Fix}
\begin{enumerate}
    \item \textbf{Stabilità}: Ridotto timestep (\texttt{rn\_Dt}) a 60 secondi per evitare instabilità CFL.
    \item \textbf{Biologia}: Disabilitato modulo PISCES (rimosso \texttt{key\_top}) per evitare crash runtime.
    \item \textbf{Output}: Abilitato \texttt{1ts} nel file XML per salvare ogni timestep.
\end{enumerate}

\section{Configurazione Specifica per Esperimento}
Di seguito le variazioni dei parametri chiave per ogni esperimento condotto.

\begin{table}[h]
\centering
\begin{tabular}{|l|c|c|l|l|}
\hline
\textbf{Exp} & \textbf{nn\_GYRE} & \textbf{rn\_Dt (s)} & \textbf{Batimetria} & \textbf{Note} \\ \hline

\textbf{A} & 1 ($\Delta x \approx 10$km) & 60 & Piatta 100m & Baseline \\ \hline
\textbf{B} & 1 ($\Delta x \approx 10$km) & 60 & Piatta 100m & $A_0 \in [0.1, 0.5, 1.0]$. Check Linearità. \\ \hline
\textbf{C} & 1 ($\Delta x \approx 10$km) & 20 & Pendenza $1000 \to 100$m & Shoaling. $zdep\_min=100m$ per stabilità. \\ \hline
\textbf{D} & 10 ($\Delta x \approx 1$km) & 5 & Piatta 100m & Altissima Risoluzione ($N_{steps}=17280$). \\ \hline
\end{tabular}
\caption{Tabella riassuntiva delle configurazioni.}
\label{tab:configs}
\end{table}

\subsection{Dettaglio Modifiche Codice}
\begin{itemize}
    \item \textbf{Exp C}: Modificato \texttt{usrdef\_zgr.F90} per introdurre pendenza lineare.
    \item \textbf{Exp D}: Modificato \texttt{namelist\_cfg} per \texttt{nn\_GYRE=10}, \texttt{rn\_Dt=5} e ripristinato \texttt{usrdef\_zgr.F90} a piatto (copia Exp B).
\end{itemize}

\end{document}
