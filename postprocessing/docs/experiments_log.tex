\documentclass[a4paper,11pt]{article}
\usepackage[utf8]{inputenc}
\usepackage{geometry}
\usepackage{graphicx}
\usepackage{float}
\usepackage{amsmath}
\usepackage{hyperref}

\geometry{margin=2.5cm}
\graphicspath{ {../} } % Path to figures (one level up in postprocessing)

\title{Diario degli Esperimenti TIDES}
\author{Filippo Di Ludovico}
\date{\today}

\begin{document}

\maketitle
\tableofcontents
\newpage

\section{Introduzione}
Questo documento raccoglie i risultati delle prove di sensibilità effettuate sul modello NEMO per il caso TIDES (propagazione Onda di Kelvin in Adriatico idealizzato).
L'obiettivo è analizzare la risposta del modello variando parametri come l'ampiezza della perturbazione iniziale.

\section{Experiment A: Baseline (Riferimento)}
\textbf{Obiettivo:} Stabilire una simulazione stabile.
\subsection{Configurazione}
\begin{itemize}
    \item \textbf{Dominio}: Bacino idealizzato rettangolare per approssimare l'Adriatico.
    \begin{itemize}
        \item Estensione: Lon $[17.75^\circ\text{E}, 19.64^\circ\text{E}]$, Lat $[40^\circ\text{N}, 49^\circ\text{N}]$.
        \item Dimensioni Griglia: $22 \times 102$ punti ($x, y$).
        \item Batimetria: Piatta costante $H = 100$ m.
        \item Risoluzione Orizzontale: $\Delta x = 10$ km (\texttt{nn\_GYRE = 1}).
        \item Risoluzione Verticale: 31 livelli.
    \end{itemize}
    \item \textbf{Perturbazione Iniziale}:
    \begin{itemize}
        \item Tipo: Anomalia Gaussiana di SSH.
        \item Ampiezza: $A_0 = 0.1$ m.
        \item Posizione: Centrata a $19.0^\circ$E, $40.5^\circ$N (Angolo Sud-Est, analogo al Canale d'Otranto).
        \item Dimensione: Raggio $\sigma = 20$ km.
    \end{itemize}
    \item \textbf{Setup Numerico}:
    \begin{itemize}
        \item Timestep: $\Delta t = 60$ s.
        \item Durata: 24 ore.
    \end{itemize}
\end{itemize}

\subsection{Risultati}
Simulazione stabile con timestep 60s.

\newpage
\subsubsection{1. Evoluzione Spaziale}
La propagazione dell'onda di Kelvin è mostrata in Figura \ref{fig:snapshotsA}. L'onda viaggia in senso antiorario lungo i bordi del bacino.

\begin{figure}[H]
    \centering
    \includegraphics[width=0.8\textwidth]{fig_expA_ssh_snapshots.png}
    \caption{Snapshots SSH per Experiment A (Baseline).}
    \label{fig:snapshotsA}
\end{figure}

\subsubsection{2. Analisi della Varianza (Punto Anfidromico)}
La mappa della varianza (Figura \ref{fig:varianceA}) mostra dove l'oscillazione è massima (antinodi, agli angoli e ai bordi) e dove è minima (al centro). La regione centrale indica il potenziale punto anfidromico.

\begin{figure}[H]
    \centering
    \includegraphics[width=0.25\textwidth]{fig_expA_ssh_variance.png}
    \caption{Mappa della Varianza SSH (Exp A). Il blu scuro al centro indica il nodo dell'onda stazionaria.}
    \label{fig:varianceA}
\end{figure}

\newpage
\subsubsection{3. Diagramma di Hovmöller}
Il percorso di estrazione per il diagramma di Hovmöller scorre lungo la costa orientale, come mostrato in Figura \ref{fig:hov_A} (sinistra).
Il diagramma lungo la costa Est (Figura \ref{fig:hov_A}, destra) mostra una propagazione lineare senza distorsioni.
\begin{figure}[H]
    \centering
    \includegraphics[width=\textwidth]{fig_expA_hovmoller.png}
    \caption{Sinistra: Percorso di estrazione (linea rossa) per il diagramma di Hovmöller. Destra: Diagramma di Hovmöller.}
    \label{fig:hov_A}
\end{figure}

\newpage
\section{Experiment B: Amplitude Sensitivity (Linearity Check)}
\textbf{Obiettivo:} Verificare la linearità della risposta del modello (SSH Input vs Output) e la dipendenza della velocità di fase dall'ampiezza.

\subsection{Configurazione}
Sono state eseguite tre run con perturbazione Gaussiana ($R=20$ km) centrata a 19.0°E, 40.5°N:
\begin{itemize}
    \item \textbf{AMP0.1}: $A_0 = 0.1$ m
    \item \textbf{AMP0.5}: $A_0 = 0.5$ m
    \item \textbf{AMP1.0}: $A_0 = 1.0$ m
\end{itemize}

\subsection{Analisi dei Risultati}

\subsubsection{1. Verifica della Linearità}
La Figura \ref{fig:linearity} mostra il confronto delle serie temporali di SSH registrate all'Estremità Nord del bacino, normalizzate per l'ampiezza iniziale $A_0$.
Si osserva che le curve normalizzate sono perfettamente sovrapposte. Questo indica che il modello risponde in modo lineare:
$$ \frac{\text{SSH}(t)}{A_0} \approx \text{costante} $$
Non si osservano sfasamenti temporali, la velocità di fase $c = \sqrt{gh}$ non è influenzata dall'ampiezza in questo regime di profondità (100m).

\begin{figure}[H]
    \centering
    \includegraphics[width=0.8\textwidth]{fig_expB_compare_linearity.png}
    \caption{Confronto delle serie temporali normalizzate all'Estremità Nord.}
    \label{fig:linearity}
\end{figure}

\noindent
Per quantificare eventuali effetti non lineari e identificare il punto anfidromico, mostriamo le analisi spaziali nelle Figure \ref{fig:nonlinear} e \ref{fig:variance}.

La mappa di non-linearità (sinistra) mostra la varianza della differenza tra i campi normalizzati:
$$ \text{Var} \left( \frac{\text{SSH}_{1.0}}{1.0} - \frac{\text{SSH}_{0.1}}{0.1} \right) $$
In un regime perfettamente lineare, questa differenza sarebbe zero ovunque. I valori osservati sono estremamente bassi (ordine $10^{-7}$ m$^2$), confermando che per ampiezze fino a 1m la dinamica è dominata dalla fisica lineare. Eventuali residui sono localizzati vicino alle coste.


La mappa a destra (Figura \ref{fig:variance}) mostra la varianza totale della SSH nel tempo.
\begin{figure}[H]
    \centering
    \begin{minipage}{0.48\textwidth}
        \centering
        \includegraphics[width=\linewidth]{fig_expB_compare_nonlinear_map.png}
        \caption{Mappa Non-linearità (Residui)}
        \label{fig:nonlinear}
    \end{minipage}\hfill
    \begin{minipage}{0.48\textwidth}
        \centering
        \includegraphics[width=\linewidth]{fig_expB_compare_amphidromic.png}
        \caption{Punto Anfidromico (Varianza SSH)}
        \label{fig:variance}
    \end{minipage}
\end{figure}

\subsection{Conclusioni}
Il sistema si comporta in modo \textbf{lineare} per perturbazioni fino a 1.0 m su fondale di 100 m. La velocità di propagazione è costante e indipendente dall'ampiezza in questo range.

\newpage
\section{Experiment C: Bathymetry Sensitivity (Pendenza Adriatica)}
\textbf{Obiettivo:} Analizzare la propagazione dell'onda di Kelvin su un fondale inclinato, simile a quello reale dell'Adriatico (da 1000m a Sud a 30m a Nord).

\subsection{Configurazione}
\begin{itemize}
    \item \textbf{Batimetria}: Pendenza lineare $H(y)$ lungo l'asse Nord-Sud.
    $$ H(y) = 1000 \text{m (Sud)} \to 100 \text{m (Nord)} $$
    
    La visualizzazione del dominio è in Figura \ref{fig:bathyC}.
    \item \textbf{Perturbazione}: $A_0 = 0.1$ m centrata a Sud-Est.
    \item \textbf{Parametri Numerici}: A causa della maggiore profondità ($H=1000$m $\Rightarrow c \approx 100$ m/s), il timestep è stato ridotto a $\Delta t = 20$ s (invece di 60 s) per soddisfare la condizione CFL. Durata totale mantenuta a 24h ($N_{steps} = 4320$).
\end{itemize}

\begin{figure}[H]
    \centering
    \includegraphics[width=0.3\textwidth]{fig_ExpC_bathymetry_3D.png}
    \caption{Batimetria Esperimento C: Pendenza lineare lungo $y$.}
    \label{fig:bathyC}
\end{figure}

\subsection{Analisi dei Risultati}

\subsubsection{1. Effetto Shoaling (Amplificazione)}
La teoria prevede che l'ampiezza dell'onda aumenti quando la profondità diminuisce. Per un canale a larghezza costante, la Legge di Green stabilisce che:
$$ A \propto H^{-1/4} $$
Confrontando la serie temporale dell'SSH all'Estremità Nord tra l'esperimento C (100m) e la Baseline (100m) in Figura \ref{fig:shoaling}, osserviamo chiaramente:
\begin{enumerate}
    \item \textbf{Amplificazione (Attesa vs Osservata)}:
    Teoricamente (Legge di Green $A \propto H^{-1/4}$), l'onda dovrebbe amplificarsi di un fattore $\approx (1000/100)^{0.25} \approx 1.77$.
    Tuttavia, osserviamo un'ampiezza minore rispetto alla Baseline. Le cause probabili sono:
    \begin{itemize}
        \item \textbf{Dispersione Numerica}: L'onda viaggia molto velocemente ($c \approx 100$ m/s) su una griglia spaziale invariata (~10 km). Questo riduce l'accuratezza nel preservare il picco.
        \item \textbf{Dinamica a Grande Scala}: A 1000m di profondità, il Raggio di Rossby ($R \approx 1000$ km) è molto maggiore della larghezza del bacino. L'onda non è più una Kelvin wave confinata, ma uno "sloshing" che coinvolge tutto il bacino, disperdendo energia su un fronte più ampio rispetto al caso a 100m.
    \end{itemize}
    \item \textbf{Velocità di Fase (Corretta)}: L'onda arriva al Nord dopo circa 4 ore, contro le 9 ore del caso a 100m. Questo è fisicamente corretto e coerente con la maggiore velocità media in acqua profonda.
\end{enumerate}

\noindent
Per visualizzare meglio l'evoluzione dell'ampiezza lungo tutto il percorso, abbiamo estratto il valore massimo dell'SSH lungo la costa orientale.
La Figura \ref{fig:shoaling_profile} mostra il confronto tra l'ampiezza osservata (blu) e quella teorica prevista dalla Legge di Green (rosso).
Si nota che l'onda parte correttamente (ampiezza $\approx 0.1$m), ma invece di crescere esponenzialmente mentre la profondità diminuisce (linea rossa), l'ampiezza osservata decade. Questo conferma che la dispersione (fisica e numerica) domina sullo shoaling in questa configurazione.

\begin{figure}[H]
    \centering
    \includegraphics[width=0.9\textwidth]{fig_ExpC_shoaling_profile.png}
    \caption{Profilo di Shoaling: Ampiezza massima lungo la costa vs Legge di Green teorica.}
    \label{fig:shoaling_profile}
\end{figure}

\begin{figure}[H]
    \centering
    \includegraphics[width=1.0\textwidth]{fig_ExpC_shoaling_comparison.png}
    \caption{Confronto SSH a Nord: Exp C (Rosso, batimetria lineare) vs Baseline (Nero, batimetria piatta). Evidente lo shoaling e lo sfasamento.}
    \label{fig:shoaling}
\end{figure}

\subsubsection{2. Diagramma di Hovmöller}
Nel diagramma spazio-tempo (Figura \ref{fig:hovmollerC}) viene sovrapposta la curva teorica del tempo di arrivo $t(y)$, calcolata integrando la velocità di fase locale (dipendente dalla profondità $H(y)$):
$$ t(y) = \int_0^y \frac{dy'}{\sqrt{g H(y')}} $$
La pendenza della traccia numerica segue la curva teorica (linea tratteggiata), confermando che l'onda viaggia alla velocità corretta data dalla batimetria. La leggera curvatura indica la decelerazione progressiva mentre l'onda risale verso acque meno profonde.

\begin{figure}[H]
    \centering
    \includegraphics[width=1.0\textwidth]{fig_ExpC_hovmoller_comparison.png}
    \caption{Confronto Diagrammi di Hovmöller. A sinistra: Exp A (velocità costante). A destra: Exp C con curva teorica sovrapposta (linea tratteggiata).}
    \label{fig:hovmollerC}
\end{figure}

\subsubsection{3. Evoluzione Spaziale (Confronto A vs C)}
Gli snapshots in Figura \ref{fig:snapshotsC} (Exp C), confrontati con quelli della Baseline (Exp A), evidenziano differenze dinamiche fondamentali:
\begin{itemize}
    \item \textbf{Velocità}: In Exp C l'onda viaggia molto più velocemente. A $T=5.5$h il fronte ha già percorso gran parte del bacino, mentre in Exp A è ancora nella prima metà.
    \item \textbf{Confinamento Costiero}: In Exp A (100m) l'onda è ben confinata alla costa (Raggio di Rossby $R \approx 300$ km). In Exp C, nelle zone profonde (1000m), $R$ aumenta drasticamente ($R \approx 1000$ km), superando la larghezza del bacino. Di conseguenza, l'onda perde il carattere di Kelvin "intrappolata" e assume un comportamento più dispersivo o di oscillazione dell'intero bacino ("sloshing").
\end{itemize}

\begin{figure}[H]
    \centering
    \includegraphics[width=0.95\textwidth]{fig_ExpC_snapshots.png}
    \caption{Snapshots Esperimento C.}
    \label{fig:snapshotsC}
\end{figure}

\newpage
\section{Experiment D: Resolution Sensitivity}
\textbf{Obiettivo:} Verificare l'impatto della risoluzione spaziale sulla dissipazione numerica.
Aumentando la risoluzione, mi aspetto che l'onda mantenga meglio la sua forma e ampiezza nel tempo, riducendo la diffusività numerica.

\subsection{Configurazione}
\begin{itemize}
    \item \textbf{Risoluzione}: GYRE = 2 ($\Delta x \approx 5$ km). Risoluzione raddoppiata rispetto alla Baseline (10 km).
    \item \textbf{Timestep}: $\Delta t = 20$ s (ridotto per CFL).
    \item \textbf{Batimetria}: Piatta 100m (come Exp A).
\end{itemize}

\subsection{Analisi dei Risultati}
\subsubsection{1. Confronto Serie Temporali}
La Figura \ref{fig:res_compare} mostra il confronto diretto dell'SSH all'Estremità Nord tra Exp A (10km) e Exp D (5km).
\begin{itemize}
    \item L'ampiezza del picco è leggermente diversa, indicando che la risoluzione influenza la conservazione dell'energia.
    \item La fase è coerente, confermando che la velocità fisica dell'onda è ben risolta in entrambi i casi.
\end{itemize}

\begin{figure}[H]
    \centering
    \includegraphics[width=0.9\textwidth]{fig_ExpD_resolution_comparison.png}
    \caption{Confronto SSH all'Estremità Nord: Baseline (Nero, 10km) vs Alta Risoluzione (Rosso, 5km).}
    \label{fig:res_compare}
\end{figure}

\subsubsection{2. Diagramma di Hovmöller}
Il diagramma spazio-tempo per il caso ad alta risoluzione (Figura \ref{fig:hov_D}) mostra dettagli più fini nella struttura dell'onda.

\begin{figure}[H]
    \centering
    \includegraphics[width=0.8\textwidth]{fig_ExpD_hovmoller.png}
    \caption{Diagramma di Hovmöller per Exp D (5km).}
    \label{fig:hov_D}
\end{figure}

\newpage
\section{Esperimento E: Validazione Fisica ($f=0$)}
\subsection{Obiettivo}
Per confermare che l'onda osservata nell'Esperimento A sia effettivamente un'onda di Kelvin costiera, abbiamo rimosso la rotazione terrestre imponendo il parametro di Coriolis $f=0$ ovunque nel dominio. Senza la forza di Coriolis, il meccanismo di intrappolamento costiero (bilancio geostrofico trasversale) viene meno.

\subsection{Risultati}
La Figura \ref{fig:expE_snap} mostra l'evoluzione dell'SSH. A differenza dell'Exp A, dove l'onda restava incollata alla costa destra, qui osserviamo:
\begin{itemize}
    \item \textbf{Propagazione Radiale}: La perturbazione si espande in tutte le direzioni come un'onda di gravità pura ("Poincaré wave" o "Gravity wave" in acqua bassa).
    \item \textbf{Simmetria}: Non c'è preferenza per il lato destro.
    \item \textbf{Decadimento}: L'energia si disperde nel bacino invece di canalizzarsi lungo la costa.
\end{itemize}
Questo conferma che il confinamento costiero osservato nella Baseline è dovuto esclusivamente alla rotazione terrestre (Onda di Kelvin).

\begin{figure}[H]
    \centering
    \includegraphics[width=0.9\textwidth]{fig_ExpE_snapshots.png}
    \caption{Snapshots Exp E ($f=0$). L'onda si propaga radialmente e simmetricamente, confermando che il confinamento costiero (Exp A) è dovuto alla rotazione.}
    \label{fig:expE_snap}
\end{figure}

\end{document}
