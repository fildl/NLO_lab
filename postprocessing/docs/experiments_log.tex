\documentclass[a4paper,11pt]{article}
\usepackage[utf8]{inputenc}
\usepackage{geometry}
\usepackage{graphicx}
\usepackage{float}
\usepackage{listings}
\usepackage{xcolor}
\usepackage{amsmath}
\usepackage{hyperref}

\geometry{margin=2.5cm}
\graphicspath{ {../} } % Path to figures (one level up in postprocessing)

\title{Diario degli Esperimenti TIDES}
\author{Filippo Di Ludovico}
\date{\today}

\begin{document}

\maketitle
\tableofcontents
\newpage

\section{Introduzione}
Questo documento raccoglie i risultati delle prove di sensibilità effettuate sul modello NEMO per il caso TIDES (propagazione Onda di Kelvin in Adriatico idealizzato).
L'obiettivo è analizzare la risposta del modello variando parametri come l'ampiezza della perturbazione iniziale.

\textbf{Nota sulle Visualizzazioni:} Per facilitare il confronto diretto tra i diversi esperimenti, tutte le mappe di SSH e i diagrammi di Hovmöller presentati in questo report utilizzano la stessa scala di colore con range $[-0.4 \text{ mm}, +0.4 \text{ mm}]$.

\section{Experiment A: Baseline (Riferimento)}
\textbf{Obiettivo:} Stabilire una simulazione stabile.
\subsection{Configurazione}
\begin{itemize}
    \item \textbf{Dominio}: Bacino idealizzato rettangolare per approssimare l'Adriatico.
    \begin{itemize}
        \item Estensione: Lon $[17.75^\circ\text{E}, 19.64^\circ\text{E}]$, Lat $[40^\circ\text{N}, 49^\circ\text{N}]$.
        \item Dimensioni Griglia: $22 \times 102$ punti ($x, y$).
        \item Batimetria: Piatta costante $H = 100$ m.
        \item Risoluzione Orizzontale: $\Delta x = 10$ km (\texttt{nn\_GYRE = 1}).
        \item Risoluzione Verticale: 31 livelli.
    \end{itemize}
    \item \textbf{Perturbazione Iniziale}:
    \begin{itemize}
        \item Tipo: Anomalia Gaussiana di SSH.
        \item Ampiezza: $A_0 = 0.1$ m.
        \item Posizione: Centrata a $19.0^\circ$E, $40.5^\circ$N (Angolo Sud-Est, analogo al Canale d'Otranto).
        \item Dimensione: Raggio $\sigma = 20$ km.
    \end{itemize}
    \item \textbf{Setup Numerico}:
    \begin{itemize}
        \item Timestep: $\Delta t = 60$ s.
        \item Durata: 24 ore.
    \end{itemize}
\end{itemize}

\subsection{Risultati}
Simulazione stabile con timestep 60s.

\newpage
\subsubsection{1. Evoluzione Spaziale}
La propagazione dell'onda di Kelvin è mostrata in Figura \ref{fig:snapshotsA}. L'onda viaggia in senso antiorario lungo i bordi del bacino.

\begin{figure}[H]
    \centering
    \includegraphics[width=0.8\textwidth]{fig_expA_ssh_snapshots.png}
    \caption{Snapshots SSH per Experiment A (Baseline).}
    \label{fig:snapshotsA}
\end{figure}

\subsubsection{2. Analisi della Varianza (Punto Anfidromico)}
La mappa della varianza (Figura \ref{fig:varianceA}) mostra dove l'oscillazione è massima (antinodi, agli angoli e ai bordi) e dove è minima (al centro). La regione centrale indica il potenziale punto anfidromico.

\begin{figure}[H]
    \centering
    \includegraphics[width=0.25\textwidth]{fig_expA_ssh_variance.png}
    \caption{Mappa della Varianza SSH (Exp A). Il blu scuro al centro indica il nodo dell'onda stazionaria.}
    \label{fig:varianceA}
\end{figure}

\newpage
\subsubsection{3. Diagramma di Hovmöller}
Il percorso di estrazione per il diagramma di Hovmöller scorre lungo la costa orientale, come mostrato in Figura \ref{fig:hov_A} (sinistra).
Il diagramma lungo la costa Est (Figura \ref{fig:hov_A}, destra) mostra una propagazione lineare senza distorsioni.
\begin{figure}[H]
    \centering
    \begin{minipage}{0.4\textwidth}
        \centering
        \includegraphics[width=\linewidth]{fig_expA_path.png}
    \end{minipage}%
    \begin{minipage}{0.6\textwidth}
        \centering
        \includegraphics[width=\linewidth]{fig_expA_hovmoller_east.png}
    \end{minipage}
    \caption{Sinistra: Percorso di estrazione (linea rossa). Destra: Diagramma di Hovmöller.}
    \label{fig:hov_A}
\end{figure}

\newpage
\section{Experiment B: Amplitude Sensitivity (Linearity Check)}
\textbf{Obiettivo:} Verificare la linearità della risposta del modello (SSH Input vs Output) e la dipendenza della velocità di fase dall'ampiezza.

\subsection{Configurazione}
Sono state eseguite tre run con perturbazione Gaussiana ($R=20$ km) centrata a 19.0°E, 40.5°N:
\begin{itemize}
    \item \textbf{AMP0.1}: $A_0 = 0.1$ m
    \item \textbf{AMP0.5}: $A_0 = 0.5$ m
    \item \textbf{AMP1.0}: $A_0 = 1.0$ m
\end{itemize}

\subsection{Analisi dei Risultati}

\subsubsection{1. Verifica della Linearità}
La Figura \ref{fig:linearity} mostra il confronto delle serie temporali di SSH registrate all'Estremità Nord del bacino, normalizzate per l'ampiezza iniziale $A_0$.
Si osserva che le curve normalizzate sono perfettamente sovrapposte. Questo indica che il modello risponde in modo lineare:
$$ \frac{\text{SSH}(t)}{A_0} \approx \text{costante} $$
Non si osservano sfasamenti temporali, la velocità di fase $c = \sqrt{gh}$ non è influenzata dall'ampiezza in questo regime di profondità (100m).

\begin{figure}[H]
    \centering
    \begin{minipage}{0.4\textwidth}
        \centering
        \includegraphics[width=\linewidth]{fig_expB_location.png}
    \end{minipage}%
    \begin{minipage}{0.6\textwidth}
        \centering
        \includegraphics[width=\linewidth]{fig_expB_compare_linearity.png}
    \end{minipage}
    \caption{Sinistra: Punto di estrazione (ross). Destra: Confronto serie temporali normalizzate.}
    \label{fig:linearity}
\end{figure}

\noindent
Per quantificare eventuali effetti non lineari e identificare il punto anfidromico, si mostrano le analisi spaziali nelle Figure \ref{fig:nonlinear} e \ref{fig:variance}.

La mappa di non-linearità (sinistra) mostra la varianza della differenza tra i campi normalizzati:
$$ \text{Var} \left( \frac{\text{SSH}_{1.0}}{1.0} - \frac{\text{SSH}_{0.1}}{0.1} \right) $$
In un regime perfettamente lineare, questa differenza sarebbe zero ovunque. I valori osservati sono estremamente bassi (ordine $10^{-7}$ m$^2$), confermando che per ampiezze fino a 1m la dinamica è dominata dalla fisica lineare. Eventuali residui sono localizzati vicino alle coste.


La mappa a destra (Figura \ref{fig:variance}) mostra la varianza totale della SSH nel tempo.
\begin{figure}[H]
    \centering
    \begin{minipage}{0.48\textwidth}
        \centering
        \includegraphics[width=\linewidth]{fig_expB_compare_nonlinear_map.png}
        \caption{Mappa Non-linearità (Residui)}
        \label{fig:nonlinear}
    \end{minipage}\hfill
    \begin{minipage}{0.48\textwidth}
        \centering
        \includegraphics[width=\linewidth]{fig_expB_compare_amphidromic.png}
        \caption{Punto Anfidromico (Varianza SSH)}
        \label{fig:variance}
    \end{minipage}
\end{figure}

\subsection{Conclusioni}
Il sistema si comporta in modo \textbf{lineare} per perturbazioni fino a 1.0 m su fondale di 100 m. La velocità di propagazione è costante e indipendente dall'ampiezza in questo range.

\newpage
\section{Experiment C: Bathymetry Sensitivity (Pendenza Adriatica)}
\textbf{Obiettivo:} Analizzare la propagazione dell'onda di Kelvin su un fondale inclinato, simile a quello reale dell'Adriatico (da 1000m a Sud a 30m a Nord).

\subsection{Configurazione}
\begin{itemize}
    \item \textbf{Batimetria}: Pendenza lineare $H(y)$ lungo l'asse Nord-Sud.
    $$ H(y) = 1000 \text{m (Sud)} \to 100 \text{m (Nord)} $$
    
    La visualizzazione del dominio è in Figura \ref{fig:bathyC}.
    \item \textbf{Perturbazione}: $A_0 = 0.1$ m centrata a Sud-Est.
    \item \textbf{Parametri Numerici}: A causa della maggiore profondità ($H=1000$m $\Rightarrow c \approx 100$ m/s), il timestep è stato ridotto a $\Delta t = 20$ s (invece di 60 s) per soddisfare la condizione CFL. Durata totale mantenuta a 24h ($N_{steps} = 4320$).
\end{itemize}

\begin{figure}[H]
    \centering
    \begin{minipage}{0.45\textwidth}
        \centering
        \includegraphics[width=\linewidth]{fig_ExpC_bathymetry_2D_1core.png}
        \caption{Mappa 2D con percorso Hovmöller.}
        \label{fig:bathyC_2D}
    \end{minipage}\hfill
    \begin{minipage}{0.5\textwidth}
        \centering
        \includegraphics[width=\linewidth]{fig_ExpC_bathymetry_3D_1core.png}
        \caption{Ricostruzione 3D (Z-coord steps).}
        \label{fig:bathyC_3D}
    \end{minipage}
    \caption{Batimetria Esperimento C: Pendenza lineare lungo $y$. La visualizzazione 3D mostra la discretizzazione a "gradini" tipica delle cordinate Z (full steps) usate nel modello.}
    \label{fig:bathyC}
\end{figure}

\subsection{Analisi della Discretizzazione Verticale}
La batimetria a "gradini" osservata in Figura \ref{fig:bathyC_3D} è intrinseca alla scelta delle \textbf{Coordinate Z (Full Steps)} ($\texttt{ln\_zco = .true.}$), dove la profondità del fondo scatta al livello verticale discreto più vicino.
Per ottenere una rappresentazione più fedele (liscia) della topografia in futuri esperimenti, si potrebbero considerare:
\begin{itemize}
    \item \textbf{Partial Steps} ($\texttt{ln\_zps = .true.}$): Mantiene le coordinate Z ma adatta lo spessore dell'ultima cella di fondo. È lo standard per minimizzare gli errori di gradiente di pressione.
    \item \textbf{Sigma Coordinates} ($\texttt{ln\_sco = .true.}$): I livelli seguono il terreno (terrain-following). Sarebbe un utile confronto per valutare come la risoluzione esatta della pendenza influenzi la propagazione dell'onda, anche se potenzialmente soggetta a errori di gradiente di pressione più elevati ("Printzelberger errors") su pendii ripidi come questo.
\end{itemize}

\subsection{Analisi dei Risultati}

\subsubsection{1. Velocità di Fase}
L'effetto della profondità variabile è evidente nella \textbf{Velocità di Fase}, come visibile nel confronto delle serie temporali al punto Nord (Figura \ref{fig:shoaling}).

\begin{figure}[H]
    \centering
    \includegraphics[width=0.8\textwidth]{fig_ExpC_shoaling_comparison_1core.png}
    \caption{Confronto SSH a Nord. Esperimento C (Rosso) vs Baseline (Nero).}
    \label{fig:shoaling}
\end{figure}

L'onda dell'Esperimento C (che ha viaggiato su fondali profondi fino a 1000m) arriva a destinazione molto prima ($\approx 4$ ore) rispetto alla Baseline (9 ore), confermando che la relazione $c = \sqrt{gH}$ è correttamente rappresentata dal modello.

\subsubsection{2. Diagramma di Hovmöller}
Nel diagramma spazio-tempo (Figura \ref{fig:hovmollerC}) viene sovrapposta la curva teorica del tempo di arrivo $t(y)$, calcolata integrando la velocità di fase locale (dipendente dalla profondità $H(y)$):
$$ t(y) = \int_0^y \frac{dy'}{\sqrt{g H(y')}} $$
La pendenza della traccia numerica segue la curva teorica (linea tratteggiata), confermando che l'onda viaggia alla velocità corretta data dalla batimetria. La leggera curvatura indica la decelerazione progressiva mentre l'onda risale verso acque meno profonde.

\begin{figure}[H]
    \centering
    \includegraphics[width=1.0\textwidth]{fig_ExpC_hovmoller_comparison_1core.png}
    \caption{Confronto Hovmöller: Exp A (Sinistra) vs Exp C (Destra).}
    \label{fig:hovmollerC}
\end{figure}

\subsubsection{3. Evoluzione Spaziale (Confronto A vs C)}
Gli snapshots in Figura \ref{fig:snapshotsC} (Exp C), confrontati con quelli della Baseline (Exp A), evidenziano differenze dinamiche fondamentali:
\begin{itemize}
    \item \textbf{Velocità}: In Exp C l'onda viaggia molto più velocemente.
    \item \textbf{Confinamento Costiero}: In Exp A (100m) l'onda è confinata alla costa (Raggio di Rossby $R = \sqrt{gH}/f \approx 300$ km). In Exp C, nelle zone profonde (1000m), $R$ aumenta ($\approx 1000$ km), superando la larghezza del bacino. Di conseguenza, l'onda perde il carattere di Kelvin "intrappolata" e assume un comportamento più dispersivo o di oscillazione dell'intero bacino.
\end{itemize}

\begin{figure}[H]
    \centering
    \includegraphics[width=0.95\textwidth]{fig_ExpC_snapshots_1core.png}
    \caption{Snapshots Esperimento C.}
    \label{fig:snapshotsC}
\end{figure}

\newpage
\section{Experiment D: Resolution Sensitivity}
\textbf{Obiettivo:} Verificare l'impatto della risoluzione spaziale sulla simulazione.

\subsection{Configurazione}
\begin{itemize}
    \item \textbf{Risoluzione}: GYRE = 2 ($\Delta x \approx 5$ km). Risoluzione raddoppiata rispetto alla Baseline (10 km).
    \item \textbf{Timestep}: $\Delta t = 20$ s (ridotto per CFL).
    \item \textbf{Batimetria}: Piatta 100m (come Exp A).
\end{itemize}

\subsection{Analisi dei Risultati}
\subsubsection{1. Confronto Serie Temporali}
La Figura \ref{fig:res_compare} mostra il confronto diretto dell'SSH all'Estremità Nord tra Exp A (10km) e Exp D (5km).
\begin{itemize}
    \item L'ampiezza del picco è leggermente diversa, indicando che la risoluzione influenza la conservazione dell'energia.
    \item La fase è coerente, confermando che la velocità fisica dell'onda è ben risolta in entrambi i casi.
\end{itemize}

\begin{figure}[H]
    \centering
    \includegraphics[width=0.9\textwidth]{fig_ExpD_resolution_comparison.png}
    \caption{Confronto SSH all'Estremità Nord: Baseline (Nero, 10km) vs Alta Risoluzione (Rosso, 5km).}
    \label{fig:res_compare}
\end{figure}

\newpage
\subsubsection{2. Diagramma di Hovmöller}
Il diagramma spazio-tempo per il caso ad alta risoluzione (Figura \ref{fig:hov_D}) mostra dettagli più fini nella struttura dell'onda.

\begin{figure}[H]
    \centering
    \includegraphics[width=0.8\textwidth]{fig_ExpD_hovmoller.png}
    \caption{Diagramma di Hovmöller per Exp D (5km).}
    \label{fig:hov_D}
\end{figure}

\newpage
\section{Esperimento E: Validazione Fisica ($f=0$)}
\subsection{Obiettivo}
Per confermare che l'onda osservata nell'Esperimento A sia effettivamente un'onda di Kelvin costiera, ho rimosso la rotazione terrestre imponendo il parametro di Coriolis $f=0$. Senza la forza di Coriolis, il meccanismo di intrappolamento costiero viene meno.

\subsection{Risultati}
La Figura \ref{fig:expE_snap} mostra l'evoluzione dell'SSH. A differenza dell'Exp A, dove l'onda restava incollata alla costa destra, qui si osserva:
\begin{itemize}
    \item \textbf{Propagazione Radiale}: La perturbazione si espande in tutte le direzioni come un'onda di gravità.
    \item \textbf{Decadimento}: L'energia si disperde nel bacino invece di canalizzarsi lungo la costa.
\end{itemize}
Questo conferma che il confinamento costiero osservato nella Baseline è dovuto esclusivamente alla rotazione terrestre (Onda di Kelvin).

\begin{figure}[H]
    \centering
    \includegraphics[width=\textwidth]{fig_ExpE_snapshots.png}
    \caption{Snapshots Exp E ($f=0$): Onde di gravità circolari, nessun confinamento.}
    \label{fig:expE_snap}
\end{figure}

\begin{figure}[H]
    \centering
    \includegraphics[width=0.8\textwidth]{fig_ExpE_hovmoller.png}
    \caption{Hovmöller Exp E: Assenza di segnale coerente lungo il bordo (l'onda si allontana).}
    \label{fig:hovE}
\end{figure}

\newpage
\appendix
\section{Guida all'Implementazione: Coordinate Sigma}
Per testare la configurazione con coordinate Sigma (e rimuovere gli "scalini" della batimetria), sono necessarie le seguenti modifiche al codice e al namelist.

\subsection{1. Modifiche a \texttt{namelist\_cfg}}
Cambiare il metodo di calcolo del gradiente di pressione idrostatico:
\begin{lstlisting}[language=Fortran, frame=single, basicstyle=\small\ttfamily]
! In &namdyn_hpg
ln_hpg_zco = .false.   ! Disabilita Z-coordinates
ln_hpg_sco = .true.    ! Abilita Sigma-coordinates
\end{lstlisting}

\subsection{2. Modifiche a \texttt{usrdef\_zgr.F90}}
È necessario ridefinire la griglia verticale per seguire il terreno.

\textbf{A. Flag Iniziali (\texttt{usr\_def\_zgr})}
Disabilitare Z-coord e abilitare Sigma:
\begin{lstlisting}[language=Fortran, frame=single, basicstyle=\small\ttfamily]
ld_zco = .FALSE.
ld_zps = .FALSE.
ld_sco = .TRUE.  ! Attiva Sigma
\end{lstlisting}

\textbf{B. Mascheramento (\texttt{zgr\_msk\_top\_bot})}
In coordinate Sigma, tutti i livelli sono bagnati fino al fondo ($k_{bot} = jpk-1$ ovunque):
\begin{lstlisting}[language=Fortran, frame=single, basicstyle=\small\ttfamily]
DO jj = 1, jpj
   DO ji = 1, jpi
      k_bot(ji,jj) = jpkm1 
   END DO
END DO
\end{lstlisting}

\textbf{C. Scaling Verticale (\texttt{zgr\_zco})}
Sostituire la logica di \texttt{zgr\_zco} per scalare i livelli 1D in base alla profondità locale ($H_{local} / H_{ref}$):
\begin{lstlisting}[language=Fortran, frame=single, basicstyle=\small\ttfamily]
! Calcolo fattore di scala
zscale = zmy_depth / 1000._wp

! Scalatura Profondita' e Spessori
pdept(ji,jj,jk) = pdept_1d(jk) * zscale
pdepw(ji,jj,jk) = pdepw_1d(jk) * zscale
pe3t (ji,jj,jk) = pe3t_1d (jk) * zscale
pe3w (ji,jj,jk) = pe3w_1d (jk) * zscale
\end{lstlisting}

\end{document}
