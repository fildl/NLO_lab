\documentclass[a4paper,11pt]{article}
\usepackage[utf8]{inputenc}
\usepackage{geometry}
\usepackage{graphicx}
\usepackage{float}
\usepackage{amsmath}
\usepackage{hyperref}

\geometry{margin=2.5cm}
\graphicspath{ {../} } % Path to figures (one level up in postprocessing)

\title{Diario degli Esperimenti TIDES}
\author{Filippo Di Ludovico}
\date{\today}

\begin{document}

\maketitle

\section{Introduzione}
Questo documento raccoglie i risultati delle prove di sensibilità effettuate sul modello NEMO per il caso TIDES (propagazione Onda di Kelvin in Adriatico idealizzato).
L'obiettivo è analizzare la risposta del modello variando parametri come l'ampiezza della perturbazione iniziale.

\section{Experiment A: Baseline (Riferimento)}
\textbf{Obiettivo:} Stabilire una simulazione stabile.
\subsection{Configurazione}
\begin{itemize}
    \item \textbf{Configurazione}: $A_0 = 1.0$m, Batimetria piatta 100m.
    \item \textbf{Estrazione}: Il percorso di estrazione per il diagramma di Hovmöller è mostrato in Figura \ref{fig:pathA}.
\end{itemize}

\begin{figure}[H]
    \centering
    \includegraphics[width=0.7\textwidth]{fig_expA_path.png}
    \caption{Percorso di estrazione (linea rossa) per il diagramma di Hovmöller (Exp A).}
    \label{fig:pathA}
\end{figure}

\subsection{Risultati}
\begin{itemize}
    \item \textbf{Risultato}: Simulazione stabile con timestep 60s.
\end{itemize}

\subsubsection{1. Evoluzione Spaziale}
La propagazione dell'onda è mostrata in Figura \ref{fig:snapshotsA}. L'onda viaggia in senso antiorario lungo i bordi del bacino (Kelvin Wave).

\begin{figure}[H]
    \centering
    \includegraphics[width=0.95\textwidth]{fig_expA_ssh_snapshots.png}
    \caption{Snapshots SSH per Experiment A (Baseline). Notare il movimento antiorario.}
    \label{fig:snapshotsA}
\end{figure}

\subsubsection{2. Analisi della Varianza (Punto Anfidromico)}
La mappa della varianza (Figura \ref{fig:varianceA}) mostra dove l'oscillazione è massima (antinodi, ai bordi) e dove è minima. Il minimo centrale indica il potenziale punto anfidromico.

\begin{figure}[H]
    \centering
    \includegraphics[width=0.7\textwidth]{fig_expA_ssh_variance.png}
    \caption{Mappa della Varianza SSH (Exp A). Il blu scuro al centro indica il nodo dell'onda stazionaria.}
    \label{fig:varianceA}
\end{figure}

\subsubsection{3. Diagramma di Hovmöller}
Il diagramma lungo la costa Est (Figura \ref{fig:hovmollerA}) mostra una propagazione lineare senza distorsioni.
\begin{figure}[H]
    \centering
    \includegraphics[width=0.8\textwidth]{fig_expA_hovmoller_east.png}
    \caption{Diagramma di Hovmöller (Exp A).}
    \label{fig:hovmollerA}
\end{figure}

\newpage
\section{Experiment B: Amplitude Sensitivity (Linearity Check)}
\textbf{Obiettivo:} Verificare la linearità della risposta del modello (SSH Output vs Input) e la dipendenza della velocità di fase dall'ampiezza.

\subsection{Configurazione}
Sono state eseguite tre run con perturbazione Gaussiana ($R=20$ km) centrata a 19.0°E, 40.5°N:
\begin{itemize}
    \item \textbf{AMP0.1}: $A_0 = 0.1$ m (Regime Lineare)
    \item \textbf{AMP0.5}: $A_0 = 0.5$ m
    \item \textbf{AMP1.0}: $A_0 = 1.0$ m (Regime Alta Ampiezza)
\end{itemize}

\subsection{Analisi dei Risultati}

\subsubsection{1. Verifica della Linearità}
Abbiamo confrontato le serie temporali dell'SSH registrate a Venezia (Nord del bacino), normalizzate per l'ampiezza iniziale $A_0$.
Come mostrato in Figura \ref{fig:linearity}, le curve normalizzate sono \textbf{perfettamente sovrapposte}. Questo indica che il modello risponde in modo lineare:
$$ \frac{\text{SSH}(t)}{A_0} \approx \text{costante} $$
Non si osservano sfasamenti temporali significativi, suggerendo che la velocità di fase $c = \sqrt{gh}$ non è influenzata dall'ampiezza in questo regime di profondità (100m).

\begin{figure}[H]
    \centering
    \includegraphics[width=0.8\textwidth]{fig_expB_compare_linearity.png}
    \caption{Confronto delle serie temporali normalizzate a Venezia. La sovrapposizione indica una risposta lineare.}
    \label{fig:linearity}
\end{figure}

Per quantificare eventuali effetti non lineari, abbiamo calcolato la mappa della varianza della differenza tra i run normalizzati (Figura \ref{fig:nonlinear}). I valori sono trascurabili, confermando la linearità.

\begin{figure}[H]
    \centering
    \includegraphics[width=0.6\textwidth]{fig_expB_compare_nonlinear_map.png}
    \caption{Mappa della non-linearità (Differenza tra run normalizzati).}
    \label{fig:nonlinear}
\end{figure}

\subsection{Conclusioni}
Il sistema si comporta in modo \textbf{lineare} per perturbazioni fino a 1.0 m su fondale di 100 m. La velocità di propagazione è costante e indipendente dall'ampiezza in questo range.

\newpage
\section{Experiment C: Bathymetry Sensitivity (Pendenza Adriatica)}
\textbf{Obiettivo:} Analizzare la propagazione dell'onda di Kelvin su un fondale inclinato, simile a quello reale dell'Adriatico (da 1000m a Sud a 30m a Nord).

\subsection{Configurazione}
\begin{itemize}
    \item \textbf{Batimetria}: Pendenza lineare $H(y)$ lungo l'asse Nord-Sud.
    $$ H(y) = 1000 \text{m (Sud)} \to 100 \text{m (Nord)} $$
    \textit{Nota: La profondità minima è stata portata a 100m (inizialmente 30m) per evitare problemi di risoluzione verticale. Con 31 livelli ($jpk=31$), ogni livello è alto circa 33m; una profondità di 30m avrebbe comportato un singolo livello attivo ("step" ripido), causando riflessioni spurie e perdita di energia.}
    
    La visualizzazione 3D del dominio è in Figura \ref{fig:bathyC}.
    \item \textbf{Perturbazione}: $A_0 = 0.1$ m (Regime Lineare) centrata a Sud-Est (Canale d'Otranto).
    \item \textbf{Parametri Numerici}: A causa della maggiore profondità ($H=1000$m $\Rightarrow c \approx 100$ m/s), il timestep è stato ridotto a $\Delta t = 20$ s (invece di 60 s) per soddisfare la condizione CFL. Durata totale mantenuta a 24h ($N_{steps} = 4320$).
\end{itemize}

\begin{figure}[H]
    \centering
    \includegraphics[width=0.9\textwidth]{fig_ExpC_bathymetry_3D.png}
    \caption{Batimetria 3D Esperimento C: Pendenza lineare lungo Y.}
    \label{fig:bathyC}
\end{figure}

\subsection{Analisi dei Risultati}

\subsubsection{1. Effetto Shoaling (Amplificazione)}
La teoria prevede che l'ampiezza dell'onda aumenti quando la profondità diminuisce (Legge di Green: $A \propto H^{-1/4}$).
Confrontando la serie temporale dell'SSH al Nord (Venezia) tra l'esperimento C (30m) e la Baseline (100m) in Figura \ref{fig:shoaling}, osserviamo chiaramente:
\begin{enumerate}
    \item \textbf{Amplificazione (Attesa vs Osservata)}:
    Teoricamente (Legge di Green $A \propto H^{-1/4}$), l'onda dovrebbe amplificarsi di un fattore $\approx (1000/100)^{0.25} \approx 1.77$.
    Tuttavia, osserviamo un'ampiezza minore rispetto alla Baseline. Le cause probabili sono:
    \begin{itemize}
        \item \textbf{Dispersione Numerica}: L'onda viaggia molto velocemente ($c \approx 100$ m/s) su una griglia spaziale invariata (~10 km). Questo riduce l'accuratezza nel preservare il picco.
        \item \textbf{Dinamica a Grande Scala}: A 1000m di profondità, il Raggio di Rossby ($R \approx 1000$ km) è molto maggiore della larghezza del bacino. L'onda non è più una Kelvin wave confinata, ma uno "sloshing" che coinvolge tutto il bacino, disperdendo energia su un fronte più ampio rispetto al caso a 100m.
    \end{itemize}
    \item \textbf{Velocità di Fase (Corretta)}: L'onda arriva al Nord dopo circa 4 ore, contro le 9 ore del caso a 100m. Questo è fisicamente corretto e coerente con la maggiore velocità media in acqua profonda.
\end{enumerate}

\begin{figure}[H]
    \centering
    \includegraphics[width=0.8\textwidth]{fig_ExpC_shoaling_comparison.png}
    \caption{Confronto SSH a Nord: Exp C (Rosso, 30m) vs Baseline (Nero, 100m). Evidente lo shoaling e il ritardo di fase.}
    \label{fig:shoaling}
\end{figure}

\subsubsection{2. Diagramma di Hovmöller e Velocità Teorica}
Il diagramma spazio-tempo (Figura \ref{fig:hovmollerC}) conferma la natura fisica della propagazione.
Abbiamo sovrapposto la curva teorica del tempo di arrivo $t(y)$, calcolata integrando la velocità di fase locale (dipendente dalla profondità $H(y)$):
$$ t(y) = \int_0^y \frac{dy'}{\sqrt{g H(y')}} $$
La pendenza della traccia numerica segue perfettamente la curva teorica (linea tratteggiata), confermando che l'onda viaggia alla velocità corretta data dalla batimetria. La curvatura indica la decelerazione progressiva mentre l'onda risale verso acque più basse.

\begin{figure}[H]
    \centering
    \includegraphics[width=1.0\textwidth]{fig_ExpC_hovmoller_comparison.png}
    \caption{Confronto Diagrammi di Hovmöller. A sinistra: Exp A (velocità costante). A destra: Exp C con curva teorica sovrapposta (linea tratteggiata).}
    \label{fig:hovmollerC}
\end{figure}

\subsubsection{3. Evoluzione Spaziale}
Gli snapshots in Figura \ref{fig:snapshotsC} mostrano l'onda che si "accumula" e cresce mentre risale il bacino.

\begin{figure}[H]
    \centering
    \includegraphics[width=0.95\textwidth]{fig_ExpC_snapshots.png}
    \caption{Snapshots Esperimento C. Notare l'aumento di intensità del colore (ampiezza) verso Nord.}
    \label{fig:snapshotsC}
\end{figure}

\end{document}
