\documentclass{beamer}

% Theme selection
\usetheme{Madrid}
\usecolortheme{default}
\setbeamertemplate{navigation symbols}{}
\setbeamertemplate{caption}[numbered]

% Packages
\usepackage[utf8]{inputenc}
\usepackage{graphicx}
\usepackage{booktabs}

% Meta-data
\title[TIDES Project]{TIDES: Dinamica dell'Onda di Kelvin \\ in un Adriatico Idealizzato}
\subtitle{Analisi di sensibilità con modello NEMO}
\author{Filippo Di Ludovico}
\institute{Università di Bologna}
\date{\today}

\begin{document}

% -----------------------------------------------------------------------------
% Title Slide
% -----------------------------------------------------------------------------
\begin{frame}
    \titlepage
\end{frame}

% -----------------------------------------------------------------------------
% Slide 1: Introduzione
% -----------------------------------------------------------------------------
\begin{frame}{Introduzione e Obiettivi}
    % TODO: Inserire contenuto
\end{frame}

% -----------------------------------------------------------------------------
% Slide 2: Exp A - Baseline Setup
% -----------------------------------------------------------------------------
\begin{frame}{Exp A: Baseline (Setup)}
    % TODO: Inserire contenuto
\end{frame}

% -----------------------------------------------------------------------------
% Slide 3: Exp A - Evoluzione Spaziale
% -----------------------------------------------------------------------------
\begin{frame}{Exp A: Evoluzione Spaziale}
    % TODO: Inserire contenuto
\end{frame}

% -----------------------------------------------------------------------------
% Slide 4: Exp A - Analisi della Propagazione
% -----------------------------------------------------------------------------
\begin{frame}{Exp A: Diagramma di Hovmöller}
    % TODO: Inserire contenuto
\end{frame}

% -----------------------------------------------------------------------------
% Slide 5: Exp A - Struttura dell'Onda (Varianza)
% -----------------------------------------------------------------------------
\begin{frame}{Exp A: Punto Anfidromico}
    % TODO: Inserire contenuto
\end{frame}

% -----------------------------------------------------------------------------
% Slide 6: Exp B - Verifica della Linearità
% -----------------------------------------------------------------------------
\begin{frame}{Exp B: Sensibilità all'Ampiezza}
    % TODO: Inserire contenuto
\end{frame}

% -----------------------------------------------------------------------------
% Slide 7: Exp C - Batimetria Variabile
% -----------------------------------------------------------------------------
\begin{frame}{Exp C: Effetto della Batimetria}
    % TODO: Inserire contenuto
\end{frame}

% -----------------------------------------------------------------------------
% Slide 8: Exp C - Risultati (Velocità di Fase)
% -----------------------------------------------------------------------------
\begin{frame}{Exp C: Confronto Velocità di Fase}
    % TODO: Inserire contenuto
\end{frame}

% -----------------------------------------------------------------------------
% Slide 9: Exp C - Deformazione e Rossby Radius
% -----------------------------------------------------------------------------
\begin{frame}{Exp C: Dinamica e Raggio di Rossby}
    % TODO: Inserire contenuto
\end{frame}

% -----------------------------------------------------------------------------
% Slide 10: Exp D - Sensibilità alla Risoluzione
% -----------------------------------------------------------------------------
\begin{frame}{Exp D: Alta Risoluzione (5km)}
    % TODO: Inserire contenuto
\end{frame}

% -----------------------------------------------------------------------------
% Slide 11: Exp E - Validazione Fisica ($f=0$)
% -----------------------------------------------------------------------------
\begin{frame}{Exp E: Ruolo della Rotazione}
    % TODO: Inserire contenuto
\end{frame}

% -----------------------------------------------------------------------------
% Slide 12: Conclusioni
% -----------------------------------------------------------------------------
\begin{frame}{Conclusioni}
    % TODO: Inserire contenuto
\end{frame}

\end{document}
