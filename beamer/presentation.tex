\documentclass{beamer}

% Theme selection
\usetheme{Madrid}
\usecolortheme{default}
\setbeamertemplate{navigation symbols}{}
\setbeamertemplate{caption}[numbered]

% Custom Colors
\definecolor{MyLightBlue}{HTML}{3993DD}
\definecolor{MyWhite}{HTML}{FFFFFF}
\definecolor{MyDarkBlue}{HTML}{0A2463}
\definecolor{MyRed}{HTML}{D8315B}
\definecolor{MyBlack}{HTML}{1E1B18}

% Apply Colors
\setbeamercolor{structure}{fg=MyDarkBlue}
\setbeamercolor{palette primary}{bg=MyDarkBlue,fg=MyWhite}
\setbeamercolor{palette secondary}{bg=MyLightBlue,fg=MyWhite}
\setbeamercolor{palette tertiary}{bg=MyRed,fg=MyWhite}
\setbeamercolor{alerted text}{fg=MyRed}
\setbeamercolor{titlelike}{parent=palette primary,fg=MyWhite}
\setbeamercolor{frametitle}{bg=MyDarkBlue,fg=MyWhite}
\setbeamercolor{background canvas}{bg=MyWhite}
\setbeamercolor{normal text}{fg=MyBlack}

% Packages
\usepackage[utf8]{inputenc}
\usepackage{graphicx}
\usepackage{booktabs}

% Meta-data
\title[TIDES]{TIDES: Dinamica dell'Onda di Kelvin \\ in un Adriatico Idealizzato}
\subtitle{Analisi di sensibilità con modello NEMO}
\author{Filippo Di Ludovico}
\institute[Università di Bologna]{Università di Bologna \\ Numerical Laboratory of Atmosphere and Ocean}
\date{14 gennaio 2026}

\begin{document}

% -----------------------------------------------------------------------------
% Title Slide
% -----------------------------------------------------------------------------
\begin{frame}
    \titlepage
\end{frame}

\begin{frame}{Introduzione e Obiettivi}
    \begin{columns}
        \begin{column}{0.65\textwidth}
            \begin{itemize}
                \item \textbf{Obiettivo del Progetto:}
                \begin{itemize}
                    \item Studiare la propagazione dell'\textbf{Onda di Kelvin} in un bacino idealizzato (analogo all'Adriatico).
                    \item Analizzare la sensibilità del \textbf{modello NEMO} a variazioni dei parametri fisici e numerici.
                \end{itemize}
                
                \vspace{0.3cm}
                
                \item \textbf{Dominio Computazionale:}
                \begin{itemize}
                    \item Griglia rettangolare $44 \times 102$ punti.
                    \item Risoluzione orizzontale $\Delta x \approx 10$ km.
                    \item \textbf{Perturbazione}: Anomalia Gaussiana ($A_0 = 0.1$ m) a Sud-Est.
                \end{itemize}
            \end{itemize}
        \end{column}
        \begin{column}{0.35\textwidth}
            \begin{figure}
                \centering
                \includegraphics[width=\textwidth]{../postprocessing/fig_expA_initial_state.png}
            \end{figure}
        \end{column}
    \end{columns}
\end{frame}

% -----------------------------------------------------------------------------
% Slide 2: Piano degli Esperimenti (TOC)
% -----------------------------------------------------------------------------
\begin{frame}{Piano degli Esperimenti}
    Panoramica delle simulazioni effettuate:
    \vspace{0.5cm}
    \begin{itemize}
        \setlength{\itemsep}{1em}
        \item \textbf{Baseline}:
        Configurazione di riferimento su fondo piatto.
        \item \textbf{Linearità}:
        Sensibilità all'ampiezza iniziale ($0.1$m, $0.5$m, $1.0$m).
        \item \textbf{Batimetria}:
        Effetto del fondale inclinato ($1000 \to 100$m).
        \item \textbf{Risoluzione}:
        Confronto tra risoluzione standard (10km) e alta (5km).
        \item \textbf{Fisica}:
        Analisi del ruolo della rotazione terrestre ($f=0$).
    \end{itemize}
\end{frame}

% -----------------------------------------------------------------------------
% Slide 3: Exp A - Baseline Setup
% -----------------------------------------------------------------------------
\begin{frame}{Exp A: Baseline \& Evoluzione}
    \begin{columns}
        \begin{column}{0.6\textwidth}
            \textbf{Configurazione:}
            \begin{itemize}
                \item Batimetria costante $H=100$ m.
                \item Timestep $\Delta t = 60$ s, Durata 24h.
            \end{itemize}
            
            \vspace{0.4cm}
            \textbf{Dinamica:}
            \begin{itemize}
                \item L'onda si propaga in senso antiorario.
                \item Rimane confinata alla costa (Onda di Kelvin).
            \end{itemize}
        \end{column}
        \begin{column}{0.4\textwidth}
            \begin{figure}
                \centering
                \includegraphics[width=\linewidth,height=0.7\textheight,keepaspectratio]{../postprocessing/fig_expA_initial_state.png}
            \end{figure}
        \end{column}
    \end{columns}
\end{frame}

% -----------------------------------------------------------------------------
% Slide 5: Exp A - Analisi della Propagazione
% -----------------------------------------------------------------------------
\begin{frame}{Exp A: Diagramma di Hovmöller}
    \begin{columns}
        \begin{column}{0.4\textwidth}
            \only<1>{
                \begin{figure}
                    \centering
                    \includegraphics[width=\linewidth,height=0.6\textheight,keepaspectratio]{../postprocessing/fig_expA_path.png}
                \end{figure}
            }%
            \only<2->{
                Il diagramma spazio-tempo lungo la costa orientale mostra una propagazione lineare senza dispersione significativa.
                
                \vspace{0.5cm}
                \textbf{Osservazioni:}
                \begin{itemize}
                     \item Velocità costante.
                     \item Fase ben preservata.
                     \item Accordo con la teoria lineare ($c = \sqrt{gH}$).
                \end{itemize}
            }
        \end{column}
        \begin{column}{0.6\textwidth}
            \begin{figure}
                \centering
                \only<1-2>{\includegraphics[width=\linewidth,height=0.85\textheight,keepaspectratio]{../postprocessing/fig_expA_hovmoller_east.png}}%
                \only<3>{\includegraphics[width=\linewidth,height=0.85\textheight,keepaspectratio]{../postprocessing/fig_expA_hovmoller_east_theory.png}}
            \end{figure}
        \end{column}
    \end{columns}
\end{frame}

% -----------------------------------------------------------------------------
% Slide 6: Exp A - Struttura dell'Onda (Varianza)
% -----------------------------------------------------------------------------
\begin{frame}{Exp A: Punto Anfidromico e Confronto}
    \begin{columns}
        \begin{column}{0.4\textwidth}
            \textbf{Risultato:} \\
            Un punto anfidromico centrale (minimo di varianza, blu).
            
            \vspace{0.3cm}
            \textbf{Letteratura:} \\
            La marea lunare semidiurna (M2) mostra la stessa struttura rotatoria con un nodo centrale.
            
            \vspace{0.3cm}
            \textit{Il modello idealizzato cattura in prima approssimazione la fisica del bacino.}
        \end{column}
        \begin{column}{0.5\textwidth}
            \begin{figure}
                \centering
                \only<1>{\includegraphics[width=0.45\linewidth,height=0.7\textheight,keepaspectratio]{../postprocessing/fig_expA_ssh_variance.png}}%
                \only<2>{\includegraphics[width=0.45\linewidth,height=0.7\textheight,keepaspectratio]{../postprocessing/fig_expA_ssh_variance_contour.png}}
                \hfill
                \includegraphics[width=0.45\linewidth,height=0.7\textheight,keepaspectratio]{../postprocessing/fig_adriatic_tides_literature.png}
            \end{figure}
        \end{column}
    \end{columns}
    \vfill
    \tiny \textit{Ref: Lovato et al., 2010}
\end{frame}

% -----------------------------------------------------------------------------
% Slide 7: Exp B - Verifica della Linearità
% -----------------------------------------------------------------------------
\begin{frame}{Exp B: Sensibilità all'Ampiezza}
    Il modello risponde linearmente aumentando l'ampiezza iniziale?
    \begin{itemize}
        \item Tre simulazioni: $A_0 = 0.1$ m, $0.5$ m, $1.0$ m.
    \end{itemize}

    \begin{columns}
        \begin{column}{0.4\textwidth}
            \begin{figure}
                \centering
                \includegraphics[height=0.5\textheight,keepaspectratio]{../postprocessing/fig_expB_location.png}
            \end{figure}
        \end{column}
        \begin{column}{0.6\textwidth}
            \begin{figure}
                \centering
                \includegraphics[height=0.5\textheight,keepaspectratio]{../postprocessing/fig_expB_compare_linearity.png}
            \end{figure}
        \end{column}
    \end{columns}
\end{frame}

% -----------------------------------------------------------------------------
% Slide 7: Exp C - Batimetria Variabile
% -----------------------------------------------------------------------------
\begin{frame}{Exp C: Effetto della Batimetria}
    Introduzione di una pendenza "tipo Adriatico":\\ $H(y)$ da 1000m (Sud) a 100m (Nord).
    
    \begin{columns}
        \begin{column}{0.5\textwidth}
            \begin{itemize}
                \item La velocità di fase teorica $c=\sqrt{gH}$ varia nello spazio.
                \item Ci aspettiamo un'onda più veloce a Sud ($c \approx 100$ m/s) che rallenta verso Nord ($c \approx 30$ m/s).
            \end{itemize}
        \end{column}
        \begin{column}{0.5\textwidth}
            \begin{figure}
                \centering
                \includegraphics[width=0.9\textwidth]{../postprocessing/fig_ExpC_bathymetry_3D_1core.png}
            \end{figure}
        \end{column}
    \end{columns}
\end{frame}

% -----------------------------------------------------------------------------
% Slide 8: Exp C - Risultati (Velocità di Fase)
% -----------------------------------------------------------------------------
\begin{frame}{Exp C: Confronto Velocità di Fase}
    \textbf{Analisi della propagazione:}
    \begin{itemize}
        \item \textbf{Sinistra:} Il confronto SSH a Nord mostra che l'onda su fondale profondo (rosso) anticipa notevolmente quella di riferimento (nera).
        \item \textbf{Destra:} Il diagramma di Hovmöller conferma che la velocità di fase segue il profilo batimetrico teorico (linea tratteggiata), decelerando verso Nord.
    \end{itemize}

    \begin{columns}
        \begin{column}{0.5\textwidth}
            \begin{figure}
                \centering
                \includegraphics[width=\linewidth,height=0.7\textheight,keepaspectratio]{../postprocessing/fig_ExpC_shoaling_comparison_1core.png}
            \end{figure}
        \end{column}
        \begin{column}{0.5\textwidth}
            \begin{figure}
                \centering
                \includegraphics[width=\linewidth,height=0.7\textheight,keepaspectratio]{../postprocessing/fig_ExpC_hovmoller_1core.png}
            \end{figure}
        \end{column}
    \end{columns}
\end{frame}

% -----------------------------------------------------------------------------
% Slide 9: Exp C - Deformazione e Rossby Radius
% -----------------------------------------------------------------------------
\begin{frame}{Exp C: Dinamica e Raggio di Rossby}
    \begin{itemize}
        \item \textbf{Exp A (100m):} $R_d \approx 300$ km. Onda confinata alla costa.
        \item \textbf{Exp C (1000m):} $R_d \approx 1000$ km. Onda a scala di bacino.
    \end{itemize}
    In acque profonde il raggio di Rossby supera la larghezza del bacino: l'onda perde il carattere costiero "stretto".
    
    \begin{figure}
        \centering
        \includegraphics[width=0.9\textwidth]{../postprocessing/fig_ExpC_snapshots_1core.png}
    \end{figure}
\end{frame}

% -----------------------------------------------------------------------------
% Slide 10: Exp D - Sensibilità alla Risoluzione
% -----------------------------------------------------------------------------
\begin{frame}{Exp D: Alta Risoluzione (5km)}
    Confronto tra $\Delta x = 10$ km (Baseline) e $\Delta x = 5$ km.
    
    \begin{figure}
        \centering
        \includegraphics[width=0.75\textwidth]{../postprocessing/fig_ExpD_resolution_comparison.png}
    \end{figure}
\end{frame}

% -----------------------------------------------------------------------------
% Slide 11: Exp E - Validazione Fisica ($f=0$)
% -----------------------------------------------------------------------------
\begin{frame}{Exp E: Ruolo della Rotazione}
    \textbf{Test:} Rimozione della forza di Coriolis ($f=0$).
    \begin{itemize}
        \item L'onda non è più intrappolata alla costa (Kelvin).
        \item Si propaga come un'onda di gravità isotropa e decade rapidamente.
    \end{itemize}

    \begin{figure}
        \centering
        \includegraphics[width=0.9\textwidth]{../postprocessing/fig_ExpE_snapshots.png}
    \end{figure}
\end{frame}

% -----------------------------------------------------------------------------
% Slide 12: Conclusioni
% -----------------------------------------------------------------------------
\begin{frame}{Conclusioni}
    \begin{enumerate}
        \setlength{\itemsep}{1em}
        \item \textbf{Dinamica Kelvin}: Il modello riproduce correttamente l'onda di Kelvin e il punto anfidromico (Exp A).
        \item \textbf{Linearità}: Risposta perfettamente lineare per ampiezze fisiche (Exp B).
        \item \textbf{Topografia}: La batimetria controlla la velocità di fase ($c=\sqrt{gH}$), con ottimo accordo teorico (Exp C).
        \item \textbf{Fisica}: Il confinamento costiero è un effetto esclusivo della rotazione (Exp E).
    \end{enumerate}
    
    \vspace{1cm}
    \centering
    \large \textbf{Grazie per l'attenzione!}
\end{frame}

\end{document}
