\documentclass{beamer}

% Theme selection
\usetheme{Madrid}
\usecolortheme{default}
\setbeamertemplate{navigation symbols}{}
\setbeamertemplate{caption}[numbered]

% Custom Colors
\definecolor{MyLightBlue}{HTML}{3993DD}
\definecolor{MyWhite}{HTML}{FFFAFF}
\definecolor{MyDarkBlue}{HTML}{0A2463}
\definecolor{MyRed}{HTML}{D8315B}
\definecolor{MyBlack}{HTML}{1E1B18}

% Apply Colors
\setbeamercolor{structure}{fg=MyDarkBlue}
\setbeamercolor{palette primary}{bg=MyDarkBlue,fg=MyWhite}
\setbeamercolor{palette secondary}{bg=MyLightBlue,fg=MyWhite}
\setbeamercolor{palette tertiary}{bg=MyRed,fg=MyWhite}
\setbeamercolor{alerted text}{fg=MyRed}
\setbeamercolor{titlelike}{parent=palette primary,fg=MyWhite}
\setbeamercolor{frametitle}{bg=MyDarkBlue,fg=MyWhite}
\setbeamercolor{background canvas}{bg=MyWhite}
\setbeamercolor{normal text}{fg=MyBlack}

% Packages
\usepackage[utf8]{inputenc}
\usepackage{graphicx}
\usepackage{booktabs}

% Meta-data
\title[TIDES Project]{TIDES: Dinamica dell'Onda di Kelvin \\ in un Adriatico Idealizzato}
\subtitle{Analisi di sensibilità con modello NEMO}
\author{Filippo Di Ludovico}
\institute{Università di Bologna}
\date{\today}

\begin{document}

% -----------------------------------------------------------------------------
% Title Slide
% -----------------------------------------------------------------------------
\begin{frame}
    \titlepage
\end{frame}

\begin{frame}{Introduzione e Obiettivi}
    \begin{columns}
        \begin{column}{0.65\textwidth}
            \begin{itemize}
                \item \textbf{Obiettivo del Progetto:}
                \begin{itemize}
                    \item Studiare la propagazione dell'\textbf{Onda di Kelvin} in un bacino idealizzato (analogo all'Adriatico).
                    \item Analizzare la sensibilità del \textbf{modello NEMO} a variazioni dei parametri fisici e numerici.
                \end{itemize}
                
                \vspace{0.3cm}
                
                \item \textbf{Dominio Computazionale:}
                \begin{itemize}
                    \item Griglia rettangolare $44 \times 102$ punti.
                    \item Risoluzione orizzontale $\Delta x \approx 10$ km.
                    \item \textbf{Perturbazione}: Anomalia Gaussiana ($A_0 = 0.1$ m) a Sud-Est.
                \end{itemize}
            \end{itemize}
        \end{column}
        \begin{column}{0.35\textwidth}
            \begin{figure}
                \centering
                \includegraphics[width=\textwidth]{../postprocessing/fig_expA_initial_state.png}
            \end{figure}
        \end{column}
    \end{columns}
\end{frame}

% -----------------------------------------------------------------------------
% Slide 1.5: Piano degli Esperimenti (TOC)
% -----------------------------------------------------------------------------
\begin{frame}{Piano degli Esperimenti}
    Panoramica delle simulazioni effettuate:
    \vspace{0.5cm}
    \begin{itemize}
        \item \textbf{Exp A (Baseline)}:
        Configurazione di riferimento su fondo piatto ($H=100$m).
        \vspace{0.2cm}
        \item \textbf{Exp B (Linearità)}:
        Sensibilità all'ampiezza iniziale ($0.1, 0.5, 1.0$m).
        \vspace{0.2cm}
        \item \textbf{Exp C (Batimetria)}:
        Effetto dello shoaling su fondo inclinato ($1000 \to 100$m).
        \vspace{0.2cm}
        \item \textbf{Exp D (Risoluzione)}:
        Confronto tra risoluzione standard (10km) e alta (5km).
        \vspace{0.2cm}
        \item \textbf{Exp E (Fisica)}:
        Analisi del ruolo della rotazione terrestre ($f=0$).
    \end{itemize}
\end{frame}

% -----------------------------------------------------------------------------
% Slide 2: Exp A - Baseline Setup
% -----------------------------------------------------------------------------
\begin{frame}{Exp A: Baseline (Setup)}
    % TODO: Inserire contenuto
\end{frame}

% -----------------------------------------------------------------------------
% Slide 3: Exp A - Evoluzione Spaziale
% -----------------------------------------------------------------------------
\begin{frame}{Exp A: Evoluzione Spaziale}
    % TODO: Inserire contenuto
\end{frame}

% -----------------------------------------------------------------------------
% Slide 4: Exp A - Analisi della Propagazione
% -----------------------------------------------------------------------------
\begin{frame}{Exp A: Diagramma di Hovmöller}
    % TODO: Inserire contenuto
\end{frame}

% -----------------------------------------------------------------------------
% Slide 5: Exp A - Struttura dell'Onda (Varianza)
% -----------------------------------------------------------------------------
\begin{frame}{Exp A: Punto Anfidromico}
    % TODO: Inserire contenuto
\end{frame}

% -----------------------------------------------------------------------------
% Slide 6: Exp B - Verifica della Linearità
% -----------------------------------------------------------------------------
\begin{frame}{Exp B: Sensibilità all'Ampiezza}
    % TODO: Inserire contenuto
\end{frame}

% -----------------------------------------------------------------------------
% Slide 7: Exp C - Batimetria Variabile
% -----------------------------------------------------------------------------
\begin{frame}{Exp C: Effetto della Batimetria}
    % TODO: Inserire contenuto
\end{frame}

% -----------------------------------------------------------------------------
% Slide 8: Exp C - Risultati (Velocità di Fase)
% -----------------------------------------------------------------------------
\begin{frame}{Exp C: Confronto Velocità di Fase}
    % TODO: Inserire contenuto
\end{frame}

% -----------------------------------------------------------------------------
% Slide 9: Exp C - Deformazione e Rossby Radius
% -----------------------------------------------------------------------------
\begin{frame}{Exp C: Dinamica e Raggio di Rossby}
    % TODO: Inserire contenuto
\end{frame}

% -----------------------------------------------------------------------------
% Slide 10: Exp D - Sensibilità alla Risoluzione
% -----------------------------------------------------------------------------
\begin{frame}{Exp D: Alta Risoluzione (5km)}
    % TODO: Inserire contenuto
\end{frame}

% -----------------------------------------------------------------------------
% Slide 11: Exp E - Validazione Fisica ($f=0$)
% -----------------------------------------------------------------------------
\begin{frame}{Exp E: Ruolo della Rotazione}
    % TODO: Inserire contenuto
\end{frame}

% -----------------------------------------------------------------------------
% Slide 12: Conclusioni
% -----------------------------------------------------------------------------
\begin{frame}{Conclusioni}
    % TODO: Inserire contenuto
\end{frame}

\end{document}
